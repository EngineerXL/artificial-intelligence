\section{Выводы}

При препроцессинге набора данных я столкнулся почти со всеми задачами, о которых я знал, но не сталкивался на практике: выбросы, сильно коррелированные признаки и перевес классов. Ручная обработка дала не очень хороший результат, поэтому я исследовал и применил методы обработки данных.

В лабораторной работе я реализовал дерево решений для регрессии, однако в нём были небольшие ошибки, и я хотел сделать реализацию более гибкой, поэтому выбрал дерево решений для классификации.

Результаты показали, что моё дерево решений показывает примерно ту же точность, что и дерево из \texttt{sklearn}, при кроссвалидации на разных общих параметрах.

Очень понравилось работать с набором данных не с \texttt{Kaggle}, где много сгенерированных данных. Я ощутил сложность обработки данных, приближенных к реальным, понял, насколько трудно бывает достичь точности даже в 50\%.
\pagebreak
