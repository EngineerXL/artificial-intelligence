\section{Задача и набор данных}
\subsection{Выбор набора данных}
Я выбрал задачу классификации и набор данных \enquote{Политическая кухня} \cite{dataset} из олимпиады \enquote{Я --- профессионал} по направлению \enquote{Искуственный интеллект (бакалавриат)}. В рамках олимпиады я участвовал в другом направлении, однако мне интересно в спокойном режиме решить задачу классификации.

\subsection{Описание}
\textit{Задача предоставлена партнером олимпиады --- Федеральным исследовательским центром "Информатика и управление" РАН}

На некотором несуществующем интернет-ресурсе \enquote{Политическая кухня} популярностью пользуются два вида видеороликов: про политику и про кулинарию. При этом под роликами про консервативную политику комментарии оставляют только консерваторы, про либеральную --- только либералы. Кулинарные видео комментируют только кулинары. Иногда на ресурсе "Политическая кухня" происходит сбой, и комментарии перепутываются (кулинарный комментарий попадает под политическое видео, либеральный --- под консервативное видео и т.п.), тогда необходимо по комментарию определить, кто его оставил (консерватор, либерал или кулинар) и перенести в соответствующий раздел. Доступа к самим текстам комментариев у команды "Политической кухни" нет, но все тексты прошли обработку лингвистическим анализатором и каждый представлен набором численных признаков.

Перед вами стоит задача разработать алгоритм машинного обучения, предсказывающий кем был написан комментарий: консерватором, либералом или кулинаром.
\pagebreak

\subsection{Формат ввода}
Тренировочная выборка \texttt{Train.csv} представляет собой csv-таблицу со столбцами-признаками и столбцом целевой переменной \texttt{target}.

Описание признаков обучающих данных:
\begin{itemize}
      \item
            \texttt{comments\_count} --- общее количество комментариев под видео, для которого создан комментарий,
      \item
            \texttt{replies\_count} --- общее количество ответов на комментарии под видео,
      \item
            \texttt{both\_count} --- общее количество сообщений под видео,
      \item
            \texttt{sentence\_count} --- общее количество предложений под видео,
      \item
            \texttt{word\_count} --- общее количество слов под видео,
      \item
            \texttt{target} - метка кем был оставлен комментарий (0 - либерал, 1 - кулинар, 2 - консерватор),
      \item
            остальные столбцы --- признаки, полученные с помощью лингвистического анализатора.
\end{itemize}
\pagebreak