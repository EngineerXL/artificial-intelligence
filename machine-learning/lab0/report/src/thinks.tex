\section{Выводы}
В ходе выполнения лабораторной работы я освежил в памяти курс математической статистики: гистограмму, корелляцию и корреляционную матрицу для наборов данных. Так же я изучил библиотеку Pandas, она оказалась очень удобной для анализа данных.

Сперва было довольно трудно выбрать задачу, которую предстоит решать. Многие наборы данных имеют много признаков, из-за этого попарные графики долго отрисовывались и было сложно по ним что-то понять.

Был проанализирован набор данных Smoker Condition \cite{kaggle}, результаты получились очень интересные: курение влияет на рак лёгких в меньшей степени, чем генетическая предрасположенность. Понятно, что выборка довольно мала, чтобы делать такой вывод. В работе этого и не требуется.
\pagebreak
