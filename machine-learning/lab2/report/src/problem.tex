\CWHeader{Лабораторная работа \textnumero 2}

\CWProblem{
Вы построили базовые (слабые) модели машинного обучения под вашу задачу. Некоторые задачи показали себя не очень, некоторые показали себя хорошо. Как выяснилось, вашим инвесторам показалось этого мало и они хотят, чтобы вы построили модели посерьезней и поточнее. Вы вспомнили, что когда то вы проходили курс машинного обучения и слышали что есть способ улучшить результаты вашей задачи: ансамбли: беггинг, пастинг, бустинг и стекинг, а также классификация путем жесткого и мягкого голосования и вы решили это опробовать. Требования к написанным классам вы оставляете теми же, что и в предыдущей работе. Будьте аккуратны в оптимизации целевой метрики и учитывайте несбалансированность классов.

\textbf{Ваша задача:}
\begin{enumerate}
    \item
    Используя модели которые вы реализовали в предыдущей лабораторной работе, реализовать два подхода для построения ансамблей: жесткое и мягкое голосование, однако учтите, некоторые модели не предусматривают оценку вероятностей, например SVM и потому вам необходимо будет оценивать вероятности;
    \item
    Реализовать дерево решений;
    \item
    Реализовать случайный лес;
    \item
    Воспользоваться готовой коробочной реализацией градиентного бустинга для решения вашей задачи.
\end{enumerate}

Для всех моделей провести fine-tuning.

}
\pagebreak
